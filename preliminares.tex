\chapter*{Presentación}

El presente Compendio Oficial reúne y sistematiza los exámenes aplicados en los
procesos de admisión de la Universidad Nacional del Altiplano – Puno durante los
años 2024 y 2025, comprendiendo tanto el Examen General como los procesos
CEPREUNA correspondientes al primer y segundo semestre de cada año.

La elaboración de este documento responde a la necesidad institucional de
contar con un registro académico organizado, que permita conservar y difundir
las evaluaciones oficiales administradas por la Dirección de Admisión (DAD),
así como brindar a los postulantes un material de estudio confiable, completo
y estructurado. 


\chapter*{Estructura del Compendio}

El contenido de este compendio se organiza en cuatro niveles claramente
definidos:

\textbf{1. Procesos de Admisión incluidos.}  
Se incorporan los exámenes oficiales de los siguientes procesos:
\begin{itemize}
	\item Examen GENERAL 2024-I y 2024-II Fase I y II
	\item Examen GENERAL 2025-I y 2025-II Fase I y II
	\item Examen CEPREUNA 2024-I y 2024-II Fase I y II
	\item Examen CEPREUNA 2025-I y 2025-II Fase I y II
\end{itemize}

\textbf{2. Clasificación por áreas académicas.}  
Los exámenes se agrupan conforme a las áreas institucionales establecidas:
\begin{itemize}
	\item Área de Ingenierías
	\item Área de Biomédicas
	\item Área de Sociales
\end{itemize}

\textbf{3. Organización por asignaturas.}  
Cada área contiene las asignaturas evaluadas oficialmente en el siguiente orden:
\begin{itemize}
	\item Aritmética, Álgebra, Geometría, Trigonometría, Física, Química
	\item Biología y Anatomía, Psicología y Filosofía, Geografía, Historia, Educación Cívic, Economía 
	\item Comunicación, Literatura, Razonamiento Matemático, Razaonamiento Verbal, Ingles, Quechua y aimara 
\end{itemize}

\textbf{4. Solucionario incluido.}  
Cada conjunto de preguntas se acompaña de su respectiva respuesta y desarrollo
cuando corresponde, permitiendo al lector revisar los procedimientos y reforzar
su aprendizaje.

\vspace{0.5cm}

Este compendio se pone a disposición de los postulantes, docentes,
especialistas y de la comunidad universitaria en general, con el propósito de
contribuir al fortalecimiento académico y a la mejora continua del proceso de
admisión.
