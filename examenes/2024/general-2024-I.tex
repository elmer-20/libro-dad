
\chapter{Examen General 2024-I}


\section*{Área: Ingenierías}


\subsection*{Aritmética}

\begin{enumerate}
	
	\item Pregunta 1 de Aritmética...
	\begin{enumerate}[label=\alph*)]
		\item A
		\item B
		\item C
		\item D
		\item E
	\end{enumerate}
	\textbf{Respuesta correcta:} c)
	
	\vspace{2mm}
	\textbf{Solucionario:}\\
	Desarrollo...
	
	\vspace{6mm}
	
	\item Pregunta 2 de Aritmética...
	\begin{enumerate}[label=\alph*)]
		\item A
		\item B
		\item C
		\item D
		\item E
	\end{enumerate}
	\textbf{Respuesta correcta:} a)
	
	\vspace{2mm}
	\textbf{Solucionario:}\\
	Desarrollo...
	
	\vspace{6mm}
	
\end{enumerate}


\subsection*{Álgebra}

\begin{enumerate}[resume]
	
	\item Pregunta 3 de Álgebra...
	\begin{enumerate}[label=\alph*)]
		\item A
		\item B
		\item C
		\item D
		\item E
	\end{enumerate}
	\textbf{Respuesta correcta:} d)
	
	\vspace{2mm}
	\textbf{Solucionario:}\\
	Explicación...
	
	\vspace{6mm}
	
	\item Pregunta 4 de Álgebra...
	\begin{enumerate}[label=\alph*)]
		\item A
		\item B
		\item C
		\item D
		\item E
	\end{enumerate}
	\textbf{Respuesta correcta:} e)
	
	\vspace{2mm}
	\textbf{Solucionario:}\\
	Explicación...
	
	\vspace{6mm}
	
\end{enumerate}


\section*{Área: Biomédicas}


\subsection*{Aritmética}

\begin{enumerate}
	
	\item Determine el esquema mólecular simplificado del siguiente circuito:
	
	
	\begin{figure}[h]
		\centering
		\includegraphics[width=0.6\textwidth]{BIO_PREG1_ARIT}
		%\caption{Descripción de la imagen}
		%\label{fig:ejemplo}
	\end{figure}
	
	
		
			
	\begin{enumerate}[label=\alph*)]
		\item $p \land q$
		\item $p \lor q$
		\item $\sim p\,\,\land \sim q$
		\item $\sim p\,\,\lor \sim q$
		\item $\sim p\,\land q$
	\end{enumerate}
	
	\textbf{Respuesta correcta:} d)
	
	\vspace{2mm}
	\textbf{Solucionario:}\\
	Del circuito inicial podemos formar lo siguiente, reduciendo:
	
	\begin{figure}[h]
		\centering
		\includegraphics[width=0.6\textwidth]{BIO_PREG1_ARIT_SOLUCION}
		%\caption{Descripción de la imagen}
		%\label{fig:ejemplo}
	\end{figure}
	
	
	De ahí tendremos la siguiente expresión lógica:
	
	\[\begin{array}{l}
		\equiv [(p \,\wedge  \sim q) \vee ( \sim p \wedge q)] \vee ( \sim p \wedge  \sim q)\\
		\equiv [(p \vee ( \sim p \wedge q)) \wedge ( \sim q \vee ( \sim p \wedge q))] \vee  \sim (p \wedge q)\\
		\equiv [(p \vee q) \wedge ( \sim q\, \vee  \sim p)] \vee  \sim (p \wedge q)\\
		\equiv ( \sim q\, \vee  \sim p) \vee ( \sim p\, \wedge  \sim q)\\
		\equiv  \sim q \vee [ \sim p \vee ( \sim p \,\wedge  \sim q)] \\ \equiv  \sim q\, \vee  \sim p\, \\ \equiv  \sim p\, \vee  \sim q
	\end{array}\]
	
	
	\vspace{6mm}
	
	\item Un cilindro de 60L de capacidad, fue llenado completamente por 4 recipientes donde el volumen del primero es al segundo como el del tercero es al cuarto como 2 es a 1. Halle la suma de los volúmenes del segundo y cuarto recipiente.
	\begin{enumerate}[label=\alph*)]
		\item 20L
		\item 30L
		\item 40L
		\item 15L
		\item 25L
	\end{enumerate}
	
	\textbf{Respuesta correcta:} a)
	
	\vspace{2mm}
	\textbf{Solucionario:}\\
	$V_1$ + $V_2$ + $V_3$ + $V_4$ = 60\\
	
	$\displaystyle\frac{V_1}{V_2} = \frac{V_3}{V_4} = \frac{2}{1}$\\
	
	$\displaystyle\frac{V_1+V_3}{V_2+V_4} = \frac{2}{1}$\\
	
	Propiedad   $\displaystyle\frac{V_1+V_2+V_3+V_4}{V_2+V_4} = \frac{2+1}{1}$\\
	
	Reemplazando   $\displaystyle\frac{60}{V_2+V_4} = \frac{3}{1}$\\
	
	Donde $V_2$ + $V_4$ = 20L
	
	\vspace{6mm}
	
	\item El dinero de A excede al de B en 20\% del dinero de C y el exceso de B a C equivaleal 10\% del dinero de A. Si A tiene S/.200 ¿Cuantos tiene C y B juntos? 
	
	
	
	
		\begin{enumerate}[label=\alph*)]
		\item 280
		\item 300
		\item 420
		\item 360
		\item 320
	\end{enumerate}
	
	\textbf{Respuesta correcta:} e)
	
	\vspace{2mm}
	\textbf{Solucionario:}\\
	
		A = S/.200 B=3 C=?\\
		
		A - B = 20\%C ...... (1)\\
		B - C = 10\%A ...... (2)\\
		A - C = 20\%C + 10\%A
		
		100\%A - 100\%C = 20\%C + 10\%A\\
		90\%A = 120\%C\\
		90\%A = 120\%C
		
		90(200) = 120C => C = S/.150
		
		En(1) 200 - B =  $\displaystyle\frac{20}{100}$(150) = S/.170 = B
		
		A + B = 150 + 170 = 320
	
	
	
	\vspace{6mm}
	
\end{enumerate}



\subsection*{Álgebra}

\begin{enumerate}
	
	\item En el polinomio P(x)=$(1+2x)^n$+$(1+3x)^n$, la suma de coeficientes excede en 23 al término independiente. Según ello, establezca el valor de verdad de las siguientes proposiciones:
	\begin{enumerate}[label=\Roman*.]
		\item El polinomio es de grado 2
		\item La suma de sus coeficientes es 25
		\item El término cuadrático es $12x^2$
	\end{enumerate}
	
	
	\begin{enumerate}[label=\alph*)]
		\item VVV
		\item VFV
		\item VVF
		\item FVV
		\item FFV
	\end{enumerate}
	
	\textbf{Respuesta correcta:} c)
	
	\vspace{2mm}
	\textbf{Solucionario:}\\
	
		$P(1)=3^n+4^n$ ; $P(0)=1+1=2$\\
		$P(1)=P(0)+23$\\
		$3^n+4^n=2+23=25$\\    
		$\therefore n = 2$\\
		$P(x)=1+4x+4x^2+1+6x+9x^2$\\
		$P(x)=13x^2+10x+2; P(1)=25$\\
		
		\begin{itemize}[label=*]
			\item El polinomio $P(x)$ es de grado 2.(V)
			\item La suma de sus coeficientes es 25 (V)
			\item El término cuadrático de $P(x)$ es $12x^2$. (F)
		\end{itemize}
	
		

	
\end{enumerate}

\begin{enumerate}[resume]
	
	\item Sabiendo que:
	${x^3} + \dfrac{1}{{{y^3}}} = {y^3} + \dfrac{1}{{{z^3}}} = 1$\\
	Calcule $(xyz)^{102}-1$
	

	\begin{enumerate}[label=\alph*)]
		\item $\,\,\,\,2$
		\item $-1$
		\item $\,\,\,\,0$
		\item $\,\,\,\,1$
		\item $-2$
	\end{enumerate}
	
	\textbf{Respuesta correcta:} c)
	
	\vspace{2mm}
	\textbf{Solucionario:}\\
	
	${x^3} + \dfrac{1}{{{y^3}}} = 1 \Rightarrow {x^3}{y^3} + 1 = {y^3}$\\
	${y^3} = 1 - \dfrac{1}{{{z^3}}}\,\,\,luego\,\,{x^3}{y^3} + 1 = 1 - \frac{1}{{{z^3}}}$\\
	${x^3}{y^3} =  - \dfrac{1}{{{z^3}}}\,\, \Rightarrow \,\,\,{x^3}{y^3}{z^3} =  - 1$\	de donde ${\left( {{x^3}{y^3}{z^3}} \right)^{34}} = {\left( { - 1} \right)^{34}} = 1$\\\\
	$\therefore {(xyz)^{102}} - 1 = 1 - 1 = 0$
	
	
\end{enumerate}


\begin{enumerate}[resume]
	
	\item Sea la función $f = [5,b] \to [a,5]$, cuya regla de correspondencia es $f(x)=x^2-6x+1$. Calcule el valor de $a+b$ siendo $f$ biyectiva. 
	\begin{enumerate}[label=\alph*)]
		\item $\sqrt{13}-1$
		\item $\sqrt{13}-2$
		\item $\sqrt{13}+2$
		\item $\sqrt{13}+1$
		\item $\sqrt{13}+3$
	\end{enumerate}
	
	\textbf{Respuesta correcta:} a)
	
	\vspace{2mm}
	\textbf{Solucionario:}\\
	
	$f(x)=x^2-6x+1=(x-3)^2-8$\\
	$f$ es creciente $f(5)=a$ y $f(b)=5$
	
	\begin{itemize}[label=.]
		\item $25 - 30 + 1 = a\Rightarrow a =  - 4$
		\item $(b + 3)^2 - 8 = 5 \Rightarrow {(b + 3)^2} = 13 \Rightarrow b = 3 + \sqrt {13}$
		
	\end{itemize}
	
	luego $a+b=\sqrt{13}-1$
	
\end{enumerate}



\subsection*{Geometría}

\begin{enumerate}
	
	\item En la figura, calcule la	$m \overset{\frown}{EF}$
	
	
	\begin{figure}[h]
		\centering
		\includegraphics[width=0.5\textwidth]{BIO_PREG1_GEOMETRIA}
		%\caption{Descripción de la imagen}
		%\label{fig:ejemplo}
	\end{figure}
	
	\begin{enumerate}[label=\alph*)]
		\item $10^\circ$
		\item $20^\circ$
		\item $35^\circ$
		\item $45^\circ$
		\item $50^\circ$
	\end{enumerate}
	
	\textbf{Respuesta correcta:} a)
	
	\vspace{2mm}
	\textbf{Solucionario:}\\
	
		De la figura:
	
	$\alpha=\dfrac{{100 - 30}}{2} = 35^\circ$\\
	$\alpha  = \dfrac{{80 - x}}{2}$\\
	
	$\to x = 80 - 2\alpha$\\
	$x=80 - 2(35)$\\
	$x = 10^\circ$
	
	\begin{tikzpicture}[overlay, remember picture]
		\node at (10, 3.5) {\includegraphics[width=0.7\textwidth]{BIO_PREG1_GEOMETRIA_SOLUCION}};
	\end{tikzpicture}
	
\end{enumerate}



\begin{enumerate}[resume]
	
	\item En la figura	PUNO y UNAP son cuadrados. Si $\alpha=16^\circ$, calcule $x$.
	
	
	\begin{figure}[h]
		\centering
		\includegraphics[width=0.45\textwidth]{BIO_PREG2_GEOMETRIA}
		%\caption{Descripción de la imagen}
		%\label{fig:ejemplo}
	\end{figure}
	
	
	\begin{enumerate}[label=\alph*)]
		\item $15^\circ$
		\item $16^\circ$
		\item $22^\circ$
		\item $31^\circ$
		\item $35^\circ$
	\end{enumerate}
	
	
	\vspace{2mm}
	\textbf{Respuesta correcta:} b)
	
	\vspace{2mm}
	\textbf{Solucionario:}\\
	
	
	
	De la figura:
	
	${45^\circ } - x + \alpha  = {45^\circ }\\
	x = \alpha \\
	x = 16^\circ$\\\\\\\\\\
	
	\begin{tikzpicture}[overlay, remember picture]
		\node at (10, 3.5) {\includegraphics[width=0.45\textwidth]{BIO_PREG2_GEOMETRIA_SOLUCION}};
	\end{tikzpicture}
	
	
\end{enumerate}


\begin{enumerate}[resume]
	
	\item Determine la ecuación de la recta que contiene al diamétro de la circunferencia $x^2+y^2-4x-6y=17$ que es perpendicular a la recta $L:5x-2y=13$.
	\begin{enumerate}[label=\alph*)]
		\item $2x+5y-19=0$
		\item $5x+2y-19=0$
		\item $2x-5y-19=0$
		\item $2x+5y+11=0$
		\item $5x+2y+11=0$
	\end{enumerate}
	
	\textbf{Respuesta correcta:} a)
	
	\vspace{2mm}
	\textbf{Solucionario:}\\
	
	$x^2+y^2-4x-6y=17\\
	{x-2}^2+{y-3}^2=17+4+9\\
	{x-2}^2+{y-3}^2=30\\
	c=(2,3)\\$
	
	$L:2y =  - 13 + 5x$
	
	$\,\,\,\,\,\,\,\,\,y =  + \frac{5}{2}x - \frac{{13}}{2} \to {m_1} =  + \frac{5}{2}\\$
	
	
	${L_p}:y - {y_0} =  - \frac{2}{5}(x - {x_0})$
		
	$\,\,\,\,\,\,\,\,\,5y + 2x + b = 0$
	
	$\,\,\,\,\,\,\,\,5(3) + 2(2) + b = 0$
	
	$\,\,\,\,\,\,\,\,b =  - 19\\$
		
	$\therefore {L_{perpendicular}} : 2x + 5y - 19 = 0$

	
	
	
\end{enumerate}


\subsection*{Trigonometría}

\begin{enumerate}
	
	\item En el gráfico, calcule el área de la región sombreada. 
	
	\begin{figure}[h]
		\centering
		\includegraphics[width=0.6\textwidth]{BIO_PREG1_TRIG}
		%\caption{Descripción de la imagen}
		%\label{fig:ejemplo}
	\end{figure}
	
	\begin{enumerate}[label=\alph*)]
		\item $\dfrac{{5\pi }}{2}{\mu ^2}$
		\item $\dfrac{{13\pi }}{2}{\mu ^2}$
		\item $\dfrac{{9\pi }}{2}{\mu ^2}$
		\item $\dfrac{{15\pi }}{2}{\mu ^2}$
		\item $\dfrac{{11\pi }}{2}{\mu ^2}$
	\end{enumerate}
	
	\textbf{Respuesta correcta:} d)
	
	\vspace{2mm}
	\textbf{Solucionario:}
	
	\begin{tikzpicture}[overlay, remember picture]
		\node at (11, 1) {\includegraphics[width=0.6\textwidth]{BIO_PREG1_TRIG_SOLUCION}};
	\end{tikzpicture}
	
	De la figura:\\
	${A_S} = \left( {\dfrac{{2\pi  + \pi }}{2}} \right)(5)\\\\
	{A_S} = \left( {\dfrac{{3\pi}}{2}} \right)(5)\\\\
	{A_S} = \dfrac{{15\pi }}{2}{\mu ^2}$
	
	
	
\end{enumerate}



\begin{enumerate}[resume]
	
	\item Si A y B son ángulos suplementarios, calcule:
	\[M=cos^2A+sen^2B\]
	\begin{enumerate}[label=\alph*)]
		\item $\,\,\,\,0$
		\item $\,\,\,\,\dfrac{1}{2}$
		\item $\,\,\,\,1$
		\item $-1$
		\item $-\dfrac{1}{2}$
	\end{enumerate}
	
	\textbf{Respuesta correcta:} c)
	
	\vspace{2mm}
	\textbf{Solucionario:}
	
	\[A+B=180^\circ\]
	\[B=180^\circ-A\]
	
	\[M=cos^2A+sen^2B\]
	\[M=cos^2A+sen^2(180^\circ-A)\]
	\[M=cos^2A+(-senA)^2\]
	\[M=cos^2A+sen^2A\]
	\[M=1\]
	
\end{enumerate}




\begin{enumerate}[resume]
	
	\item En el gráfico, AB$=6$; BC$=5$ y AC$=4$
	
	
	\begin{figure}[h]
		\centering
		\includegraphics[width=0.6\textwidth]{BIO_PREG3_TRIG}
		%\caption{Descripción de la imagen}
		%\label{fig:ejemplo}
	\end{figure}
	
	
	Calcule:
	\[M = \frac{{sen\alpha  + sen\beta }}{{sen\theta }}\]
	
	\begin{enumerate}[label=\alph*)]
		\item $\dfrac{4}{3}$
		\item $\dfrac{3}{2}$
		\item $\dfrac{3}{4}$
		\item $\dfrac{2}{3}$
		\item $\dfrac{2}{5}$
	\end{enumerate}
	
	\textbf{Respuesta correcta:} b)
	
	\vspace{2mm}
	\textbf{Solucionario:}\\
	
	Ley de senos:
	
	$M = \dfrac{{4}}{{sen\beta }}+\dfrac{{5}}{{sen\alpha }}+\dfrac{{6}}{{sen\theta }}=2R\\\\
	M=\dfrac{{sen\beta}}{{4}}+\dfrac{{sen\alpha}}{{5}}+\dfrac{{sen\theta}}{{6}}=\dfrac{{1}}{{2R}}$
	
	
	
	
	
	$ \left\{ {\begin{array}{*{20}{c}}
			sen\beta=\dfrac{{4}}{{2R}}\\\\
			sen\alpha=\dfrac{{5}}{{2R}}\\\\
			sen\theta=\dfrac{{6}}{{2R}}
	\end{array}} \right.$
	
	$M = \dfrac{{\dfrac{5}{{2R}} + \dfrac{4}{{2R}}}}{{\dfrac{6}{{2R}}}} = \dfrac{{\dfrac{9}{{2R}}}}{{\dfrac{6}{{2R}}}} = \dfrac{9}{6} = \dfrac{3}{2}$
	
	\begin{tikzpicture}[overlay, remember picture]
		\node at (11.5, 3) {\includegraphics[width=0.6\textwidth]{BIO_PREG3_TRIG_SOLUCION}};
	\end{tikzpicture}
	
	
	
\end{enumerate}



\subsection*{Física}

\begin{enumerate}
	
	\item La figura muestra un cubo de arista $x$. Determine módulo de la resultante de los vectores  $\vec{R_1}$ y $\vec{R_2}$.
	
	
	\begin{figure}[h]
		\centering
		\includegraphics[width=0.4\textwidth]{BIO_PREG1_FISICA}
		%\caption{Descripción de la imagen}
		%\label{fig:ejemplo}
	\end{figure}
	
	
	
	\begin{enumerate}[label=\alph*)]
		\item $4x$
		\item $\,\,\,x$
		\item $2x$
		\item $3x$
		\item $5x$
	\end{enumerate}
	
	
	
	
	%\begin{tikzpicture}[overlay, remember picture]
	%	\node at (11, 3) %{\includegraphics[width=0.4\textwidth]{BIO_PREG1_FISICA_SOLUCION}};
	%\end{tikzpicture}
	
	
	\textbf{Respuesta correcta:} d)
	
	\vspace{2mm}
	\textbf{Solucionario:}\\
	
	
	
	
	%\AddToShipoutPicture*{\put(100,80){\includegraphics[scale=0.8]{BIO_PREG1_FISICA_SOLUCION}}} % Image background

	
	%\begin{figure}[h]
	%	\centering
	%	\includegraphics[width=0.4\textwidth]{BIO_PREG1_FISICA_SOLUCION}
		%\AddToShipoutPicture*{\put(100,80){\includegraphics[scale=0.8]{BIO_PREG1_FISICA_SOLUCION}}} % Image background
		%\caption{Descripción de la imagen}
		%\label{fig:ejemplo}
	%\end{figure}
	
		
	De la figura:
	
	$A = (x,x,0)\\
	B = (0,0,x)\\
	C = (0,x,x)$
	
	
	$\overrightarrow {{R_{1}}}  = \overrightarrow {AB}  = B - A = (0,0,x) - (x,x,0)\\
	\overrightarrow {{R_{1}}}  = ( - x, - x,x)\\
	\\$
	
	$\overrightarrow {{R_{2}}}  = \overrightarrow {AC}  = C - A = (0,x,x) - (x,x,0)\\
	\overrightarrow {{R_{2}}}  = ( - x,0,x)\\
	\\$
	
	$\overrightarrow R  = \overrightarrow {{R_{1}}}  + \overrightarrow {{R_{2}}} \\
	\overrightarrow R  = ( - x, - x,x) + ( - x,0,x)\\
	\overrightarrow R  = ( - 2x, - x,2x)$
	
	$\left| {\overrightarrow R } \right| = \sqrt {{{( - 2x)}^2} + {{( - x)}^2} + {{(2x)}^2}}\\\\
	\left| {\overrightarrow R } \right| = \sqrt {{{4x}^2} + {{x}^2} + {{4x}^2}}\\
	\left| {\overrightarrow R } \right| = \sqrt {{{9x}^2}}=3x$	
	
	\begin{tikzpicture}[overlay, remember picture]
		\node at (11, 3) {\includegraphics[width=0.4\textwidth]{BIO_PREG1_FISICA_SOLUCION}};
	\end{tikzpicture}
	
	
	
\end{enumerate}




\begin{enumerate}[resume]
	
	\item Un cuerpo esférico se lanza contra la pared, como se muestra en la figura. Si luego del impacto su rapidez disminuye en $1/3$, determine la relación entre la energía cinética antes y después del impacto. 
	
	
	\begin{figure}[h]
		\centering
		\includegraphics[width=0.3\textwidth]{BIO_PREG2_FISICA}
		%\caption{Descripción de la imagen}
		%\label{fig:ejemplo}
	\end{figure}
	
	
	
	\begin{enumerate}[label=\alph*)]
		\item $9/2$
		\item $9/4$
		\item $9/5$
		\item $\,\,\,\,\,9$
		\item $\,\,\,\,\,4$
	\end{enumerate}
	
	\textbf{Respuesta correcta:} b)
	
	\vspace{2mm}
	\textbf{Solucionario:}\\
	
	
	$E_{c}\,\text{antes}=\frac{1}{2}mv^2\\$
	
	$E_{c}\,\text{después}=\frac{1}{2}m({{\frac{3}{3}v-\frac{1}{3}v}})^2\\\\$
	$E_{c}\,\text{después}=\frac{1}{2}m({\frac{2}{3}v})^2\\\\$
	$E_{c}\,\text{después}=\frac{1}{2}m{\frac{4}{2}v}^2=\frac{2}{9}mv^2\\$
	
	$\dfrac{{{E_c}\,\text{antes}}}{{{E_c}\,\text{después}}} = \dfrac{{\frac{1}{2}m{v^2}}}{{\frac{2}{9}m{v^2}}} = \dfrac{9}{4}$
	
	
\end{enumerate}



\begin{enumerate}[resume]
	
	\item Se tiene un alambre largo y delgado por el cual fluye una corriente de $20$ A. ¿A qué distancia del conductor la magnitud del campo magnético es $5\times10^{-4}T$? 
	\newline
	$\mu_o=4\pi\times10^{-7}\,\frac{Tm}{\text{A}}$  
		
		
	\begin{enumerate}[label=\alph*)]
		\item $0,4$ cm
		\item $0,5$ cm
		\item $0,6$ cm
		\item $0,7$ cm
		\item $0,8$ cm
	\end{enumerate}
	
	\textbf{Respuesta correcta:} e)
	
	\vspace{2mm}
	\textbf{Solucionario:}\\
	
	%\begin{figure}[h]
	%	\centering
	%	\includegraphics[width=0.23\textwidth]{PREG3_FISICA_SOLUCION}
		%\caption{Descripción de la imagen}
		%\label{fig:ejemplo}
	%\end{figure}
	
	
	$B = \dfrac{{{\mu_o}i}}{{2\pi x}}\\\\
	x = \dfrac{{{\mu_o}i}}{{2\pi B}}\\\\
	x = \dfrac{{4\pi  \times {{10}^{ - 7}}\frac{{Tm}}{A}(20A)}}{{2\pi (5\times {{10}^{ - 4}}T)}}\\
	x = \dfrac{{4\times {{10}^{ - 6}}}}{{5\times{{10}^{ - 4}}}} = \dfrac{4}{5}\times{10^{ - 2}}m\\
	x = 0,8\,cm$
	
	\begin{tikzpicture}[overlay, remember picture]
		\node at (11, 3) {\includegraphics[width=0.23\textwidth]{PREG3_FISICA_SOLUCION}};
	\end{tikzpicture}
	
	
\end{enumerate}



\subsection*{Química}

\begin{enumerate}
	
	\item ¿Cuál es la normalidad de una solución que contiene $50$g de ácido sulfúrico en $500$mL de solución?
	\newline
	$(\text{H}=1;\,\text{S}=32;\,\text{O}=16)$
	
	\begin{enumerate}[label=\alph*)]
		\item $2,00$
		\item $2,08$
		\item $2,04$
		\item $2,40$
		\item $2,50$
	\end{enumerate}
	
	\textbf{Respuesta correcta:} c)
	
	\vspace{2mm}
	\textbf{Solucionario:}\\
	
	Datos: (H$_2$SO$_4$)
	
	$\text{H}=1\times2\,\,\,=2\\
	\text{S}\,=32\times1=32\\
	\text{O}=16\times4=64$
	
	$\,\,\,\,\,\,\,\,\,\,\,\,\,\,\,\,\,\,\,\,\,\,\,\,\,\,\,\,\,\,=\overline {{\rm{98 g/mol}}}\\$
	
	Volumen solución $=500\text{ml}=0.5\text{L}\\$
	Gramos de soluto $=50\text{g}$\\
	
	Hallando el número equivalente gramo de un ácido:
		
	${\rm{Eq}} - {\rm{g}} = \dfrac{{{\rm{gr}}\,{\rm{soluto}}}}{{{\rm{\# }}\,{\rm{Equivalente}}\,{\rm{gramo}}\,{\rm{\text{ácido}}}\,{\rm{(H)}}}}\\\\
	{\rm{Eq}} - {\rm{g}}\,\,{\rm{\text{ácido}}}\, = \dfrac{{{\rm{gr}}\,{\rm{soluto}}}}{{{\rm{\# }}\,{\rm{Equivalente}}\,{\rm{gramo}}\,{\rm{\text{ácido}}}\,{\rm{(H)}}}}\\\\
	{\rm{Eq}} - {\rm{g}}\,\,{\rm{\text{ácido}}}\,{\rm{\text{sulfúrico}}} = \,49\,{\rm{g}}$\\
	
	Hallando el número equivalente:\\
	${\rm{Eq}} - {\rm{g}} = \dfrac{{50\,{\rm{g}}}}{{49\,{\rm{g}}}} = 1,02\,{\rm{g}}$
	
	Hallando la normalidad:\\
	
	${\rm{N = }}\dfrac{{{\rm{N^\circ  Eq}} - {\rm{g}}}}{{{\rm{U}}\,{\rm{\text{solución}}}}}\\\\
	{\rm{N = }}\dfrac{{1,02}}{{{\rm{0,5}}}}\\\\
	{\rm{N = 2,04}}$
	
	
	
\end{enumerate}


\begin{enumerate}[resume]
	
	\item En la relación a la ecuación química, indique verdadero (V) O (F) según corresponda.
	\[HC{I_{(ac)}} + NaO{H_{(ac)}} \to {H_2}{O_{(l)}} + calor\]
	
	\begin{enumerate}[label=\Roman*.]
		\item Es una reacción de metátesis.
		\item Es una reacción exotérmica.
		\item Es una reacción de simple desplazamiento.
	\end{enumerate}
	
	
	\begin{enumerate}[label=\alph*)]
		\item VVV
		\item FVF
		\item FVV
		\item VVF
		\item FFV
	\end{enumerate}
	
	\textbf{Respuesta correcta:} d)
	
	\vspace{2mm}
	\textbf{Solucionario:}\\
	
	
	\begin{enumerate}[label=\Roman*.]
		\item ${\rm{Verdadero}} \to \text{Posee la forma}\,\,\,\,\,\,\,\,{\rm{AB}} + {\rm{CD}} \to {\rm{AC}} + {\rm{BD}}\\
		{\rm{   }} \Rightarrow {\text{corresponde a una reacción de metátesis. }}$
		
		\item ${\rm{Verdadero}} \to \text{Como la reacción genera calor}\\
		{\rm{   }} \Rightarrow {\text{es una reacción exotérmica. }}$
		
		\item ${\rm{Falso}} \to \text{Al ser una reacción de metátesis se desarrolla doble desplazamiento.}$\\
		
	\end{enumerate}
	
\end{enumerate}


\begin{enumerate}[resume]
	
	\item Se necesitan $200$ calorías para que $500$g de un cuerpo eleve su temperatura desde $20^\circ$C hasta $25^\circ$C. Determine el peso atómico de este cuerpo. 
	\begin{enumerate}[label=\alph*)]
		\item $64,5$
		\item $78,7$
		\item $36,4$
		\item $86,5$
		\item $49,5$
	\end{enumerate}
	
	\textbf{Respuesta correcta:} b)
	
	\vspace{2mm}
	\textbf{Solucionario:}\\
	
	Datos:\\
	$d=200\,cal$\\
	$m=500 \rm{g}$\\
	$\bigtriangleup T=25-20=5^\circ C$
	
	Según la ley de Dulong y Petit:\\
	$PA \times CE = 6,3\\
	PA = \dfrac{{6,3}}{{CE}}$\\
	
	$d=m\times CE\times \bigtriangleup T\\\\
	CE = \dfrac{{d}}{{m\times \bigtriangleup T}}\\\\
	CE = \dfrac{{200\,cal}}{{500 \rm{g}\times 20^\circ C}}=\dfrac{2}{25}$\\\\
	
	Reemplazando:\\\\
	$PA = \dfrac{{6,3}}{{2/25}} = \dfrac{{25 \times 6,3}}{2} = 25 \times 3,15\\\\
	\boxed{{\therefore PA = 78,7}}$
	
	
\end{enumerate}


\begin{enumerate}[resume]
	
	\item ¿Qué procesos no están relacionados con la electrólisis?
	
		\begin{enumerate}[label=\Roman*.]
		\item La producción de energía eléctrica
		\item La producción de metales con una fina capa de otro metal
		\item La descarga de la batería en los teléfonos móviles
		\item La síntesis de elementos 
		
	\end{enumerate}
	
	
	\begin{enumerate}[label=\alph*)]
		\item Sólo II
		\item Sólo I
		\item Sólo IV
		\item I y III
		\item IV, II y III
	\end{enumerate}
	
	\textbf{Respuesta correcta:} d)
	
	\vspace{2mm}
	\textbf{Solucionario:}
	
	ELECTROLÍSIS\\
	Es un proceso no espontáneo en el cual, por acción de la energía eléctrica que proviene de una fuente de voltaje, se logra desarrollar reacciones REDOX. Se realiza en un medio acuoso. Por lo tanto la producción de energía eléctrica y la descarga de la batería en los telefonos móviles, no  se realiza en medio acuoso.
	
	
\end{enumerate}


\begin{enumerate}[resume]
	
	\item Un gas está confinado en un cilindro de acero de $10,0$ litros, a un apresión de 5 atmósferas y a $27^\circ$C. Si la temperatura se eleva a  $327^\circ$C. ¿Cuál será la presión del gas (en atm) en el interior del cilindro?
	
	\begin{enumerate}[label=\alph*)]
		\item $\,\,\,1,5$
		\item $10,0$
		\item $\,\,\,7,5$
		\item $12,5$
		\item $15,0$
	\end{enumerate}
	
	\textbf{Respuesta correcta:} a)
	
	\vspace{2mm}
	\textbf{Solucionario:}\\
	
	
	%\iffalse
	\begin{minipage}[t]{0.45\textwidth}
		Datos:\\
		$V = 10\,L\\
		{P_{\,1}} = 5\,atm\\
		{T_{\,1}} = 27^\circ \rm{C}\\
		{T_{\,2}} = 327^\circ \rm{C}\\
		{P_{\,2}} = ?$
	\end{minipage}
	\hfill
	\begin{minipage}[t]{0.45\textwidth}
		$\dfrac{{{P_{\,1}}}}{{{T_{\,1}}}} = \dfrac{{{P_{\,2}}}}{{{T_{\,2}}}}\\\\
		{P_{\,2}} = \dfrac{{{T_{\,2}}}}{{{T_{\,1}}}}{P_{\,1}}\\\\
		{P_{\,2}} = \dfrac{{327^\circ \rm{C} + 273}}{{27^\circ \rm{C} + 273}}(5\,atm)\\\\
		{P_{\,2}} = \dfrac{{600}}{{300}}(5\,atm)\\\\
		\boxed{{P_{\,2}} = 10\,atm}$
	\end{minipage}
	%\fi
	
\end{enumerate}