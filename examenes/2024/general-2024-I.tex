
\chapter{Examen General 2024-I}


\section*{Área: Ingenierías}


\subsection*{Aritmética}

\begin{enumerate}
	
	\item Pregunta 1 de Aritmética...
	\begin{enumerate}[label=\alph*)]
		\item A
		\item B
		\item C
		\item D
		\item E
	\end{enumerate}
	\textbf{Respuesta correcta:} c)
	
	\vspace{2mm}
	\textbf{Solucionario:}\\
	Desarrollo...
	
	\vspace{6mm}
	
	\item Pregunta 2 de Aritmética...
	\begin{enumerate}[label=\alph*)]
		\item A
		\item B
		\item C
		\item D
		\item E
	\end{enumerate}
	\textbf{Respuesta correcta:} a)
	
	\vspace{2mm}
	\textbf{Solucionario:}\\
	Desarrollo...
	
	\vspace{6mm}
	
\end{enumerate}


\subsection*{Álgebra}

\begin{enumerate}[resume]
	
	\item Pregunta 3 de Álgebra...
	\begin{enumerate}[label=\alph*)]
		\item A
		\item B
		\item C
		\item D
		\item E
	\end{enumerate}
	\textbf{Respuesta correcta:} d)
	
	\vspace{2mm}
	\textbf{Solucionario:}\\
	Explicación...
	
	\vspace{6mm}
	
	\item Pregunta 4 de Álgebra...
	\begin{enumerate}[label=\alph*)]
		\item A
		\item B
		\item C
		\item D
		\item E
	\end{enumerate}
	\textbf{Respuesta correcta:} e)
	
	\vspace{2mm}
	\textbf{Solucionario:}\\
	Explicación...
	
	\vspace{6mm}
	
\end{enumerate}


\section*{Área: Biomédicas}


\subsection*{Aritmética}

\begin{enumerate}
	
	\item Determine el esquema mólecular simplificado del siguiente circuito:
	
	\[
	\begin{array}{c}
		\text{Circuito:}\\[4pt]
		\left[
		\begin{array}{c}
			p \rightarrow q \\[-2pt]
			\neg p \rightarrow q \\[-2pt]
			\neg p \rightarrow \neg q
		\end{array}
		\right]
	\end{array}
	\]
	
	
	
		
			
	\begin{enumerate}[label=\alph*)]
		\item p $\land$ q
		\item p $\lor$ q
		\item $\sim$ p $\land$ $\sim$ q
		\item $\sim$ p $\lor$ $\sim$ q
		\item $\sim$ p $\land$ q
	\end{enumerate}
	
	\textbf{Respuesta correcta:} d)
	
	\vspace{2mm}
	\textbf{Solucionario:}\\
	Del circuito inicial podemos formar lo siguiente, reduciendo:
	
	\[
	\left[
	\begin{array}{c}
		p \land \neg q \\[-2pt]
		\neg p \land q \\[-2pt]
		\neg p \land \neg q
	\end{array}
	\right]
	\]
	
	De ahí tendremos la siguiente expresión lógica:
		
	\[
	\begin{aligned}
		&\equiv [(p \land \neg q) \lor (\neg p \land q)] \lor (\neg p \land \neg q) \\[6pt]
		&\equiv \big[(p \lor \neg p)(\neg q \lor q)\big] \lor (\neg p \land \neg q) \\[6pt]
		&\equiv (p \lor q) \land (\neg q \lor p) \lor (\neg p \land q) \\[6pt]
		&\equiv (\neg q \lor p) \lor (\neg p \land q) \\[6pt]
		&\equiv \neg q \lor [\neg p \lor (p \land \neg q)] \\[6pt]
		&\equiv \neg q \lor p \\[6pt]
		&\equiv \neg p \lor \neg q
	\end{aligned}
	\]
	
	\vspace{6mm}
	
	\item Un cilindro de 60L de capacidad, fue llenado completamente por 4 recipientes donde el volumen del primero es al segundo como el del tercero es al cuarto como 2 es a 1. Halle la suma de los volúmenes del segundo y cuarto recipiente.
	\begin{enumerate}[label=\alph*)]
		\item 20L
		\item 30L
		\item 40L
		\item 15L
		\item 25L
	\end{enumerate}
	
	\textbf{Respuesta correcta:} a)
	
	\vspace{2mm}
	\textbf{Solucionario:}\\
	$V_1$ + $V_2$ + $V_3$ + $V_4$ = 60\\
	
	$\displaystyle\frac{V_1}{V_2} = \frac{V_3}{V_4} = \frac{2}{1}$\\
	
	$\displaystyle\frac{V_1+V_3}{V_2+V_4} = \frac{2}{1}$\\
	
	Propiedad   $\displaystyle\frac{V_1+V_2+V_3+V_4}{V_2+V_4} = \frac{2+1}{1}$\\
	
	Reemplazando   $\displaystyle\frac{60}{V_2+V_4} = \frac{3}{1}$\\
	
	Donde $V_2$ + $V_4$ = 20L
	
	\vspace{6mm}
	
	\item El dinero de A excede al de B en 20\% del dinero de C y el exceso de B a C equivaleal 10\% del dinero de A. Si A tiene S/.200 ¿Cuantos tiene C y B juntos? 
	
	
	
	
		\begin{enumerate}[label=\alph*)]
		\item 280
		\item 300
		\item 420
		\item 360
		\item 320
	\end{enumerate}
	
	\textbf{Respuesta correcta:} e)
	
	\vspace{2mm}
	\textbf{Solucionario:}\\
	
		A = S/.200 B=3 C=?\\
		
		A - B = 20\%C ...... (1)\\
		B - C = 10\%A ...... (2)\\
		A - C = 20\%C + 10\%A
		
		100\%A - 100\%C = 20\%C + 10\%A\\
		90\%A = 120\%C\\
		90\%A = 120\%C
		
		90(200) = 120C => C = S/.150
		
		En(1) 200 - B =  $\displaystyle\frac{20}{100}$(150) = S/.170 = B
		
		A + B = 150 + 170 = 320
	
	
	
	\vspace{6mm}
	
\end{enumerate}



\subsection*{Álgebra}

\begin{enumerate}[resume]
	
	\item En el polinomio P(x)=$(1+2x)^n$+$(1+3x)^n$, la suma de coeficientes excede en 23 al término independiente. Según ello, establezca el valor de verdad de las siguientes proposiciones:
	\begin{enumerate}[label=\Roman*.]
		\item El polinomio es de grado 2
		\item La suma de sus coeficientes es 25
		\item El término cuadrático es $12x^2$
	\end{enumerate}
	
	
	\begin{enumerate}[label=\alph*)]
		\item VVV
		\item VFV
		\item VVF
		\item FVV
		\item FFV
	\end{enumerate}
	
	\textbf{Respuesta correcta:} a)
	
	\vspace{2mm}
	\textbf{Solucionario:}\\
	P(1)=$3^n$+$4^n$ ; $P(0)$=1+1=2\\
	P(1)=P(0)+23\\
	$3^n$+$4^n$=2+23=25    
	
\end{enumerate}

\begin{enumerate}[resume]
	
	\item Tercera pregunta en esta área...
	\begin{enumerate}[label=\alph*)]
		\item A
		\item B
		\item C
		\item D
		\item E
	\end{enumerate}
	
	\textbf{Respuesta correcta:} a)
	
	\vspace{2mm}
	\textbf{Solucionario:}\\
	Explicación...
	
\end{enumerate}


\subsection*{Química}

\begin{enumerate}[resume]
	
	\item Tercera pregunta en esta área...
	\begin{enumerate}[label=\alph*)]
		\item A
		\item B
		\item C
		\item D
		\item E
	\end{enumerate}
	
	\textbf{Respuesta correcta:} a)
	
	\vspace{2mm}
	\textbf{Solucionario:}\\
	Explicación...
	
\end{enumerate}
